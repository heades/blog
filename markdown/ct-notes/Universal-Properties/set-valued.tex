A functor $F : \cat{C} \mto \sets$ from some category $\cat{C}$ to the
category of sets is called a \textbf{set-valued functor}.  It relates
objects to sets and morphisms to functions.  One can think of them as
an interpretation of an abstract category to a concrete category; that
is, we can interpret a category without a notion of elements into one
with such a notion, and one that we understand quite well.

We give several example set-valued functors below.
\begin{enumerate}  
\item
  Suppose $\cat{C}$ has objects $\{0,1\}$ and the following
  morphisms:
  \begin{center}
    \begin{math}
      1 \two<500> 0
    \end{math}
  \end{center}
  Then a functor $F : \cat{C} \to \sets$ maps $1$ to a set $E$ and $0$
  to a set $N$, and the two arrows to functions $s : E \to N$ and $t :
  E \to N$.  A graph is a tuple $(N,E,s,t)$ where $N$ is the set of
  nodes, $E$ is the set of edge labels, $s : E \to N$ assigns edges
  their source nodes, and $t : E \to N$ assigns edges their target
  nodes.  Therefore, $F$ represents a graph!  In addition, natural
  transformations between functors like $F$ correspond to graph
  homomorphisms \cite{awodey2006category}.  Notice that this functor
  is definable for any category at all, and thus, we can think of a
  category as an abstraction of a directed graph with a monoid
  structure given by composition.

\item A monoid $([[M]], [[e]], [[(x)]])$ can be thought of as a set-valued functor
  $[[M : 1 -> Set]]$ with two natural transformations:
  \[
  \begin{array}{lll}
    \text{(Identity)} & [[e : {*} -> M]]\\
    \text{(Multiplication)} & [[ [(x)] : M * M -> M]]
  \end{array}
  \]
  with the following commutative diagrams:
  \[
  \begin{array}{c}
    \bfig
    \qtriangle|ama|/<-`->`->/<1000,400>[ [[M * M]]`[[{*} * M]]`[[M]];[[e * id M]]`[[ [(x)] ]]`[[pi2]] ]
    \btriangle|ama|/<-`->`->/<1000,400>[ [[M * M]]`[[M * {*}]]`[[M]];[[id M * e]]`[[ [(x)] ]]`[[pi1]] ]
    \efig
    \\[5pt]    
    \bfig
    \square|amma|/->`=`->`/<1900,400>[ [[(M * M) * M]]`[[M * M]]`[[M * (M * M)]]`[[M]];[[ [(x)] * id M]]``[[ [(x)] ]]`]
    \morphism<950,0>[ [[M * (M * M)]]`[[M * M]];[[id M * [(x)] ]] ]
    \morphism(950,0)<950,0>[ [[M * M]]`[[M]];[[ [(x)] ]] ]
    \efig
  \end{array}
  \]
  The left diagram corresponds to the identity axioms of the monoid,
  and the right diagram corresponds to associativity of
  multiplication.
  In this example, we made use of the following functors:
  \[
  \begin{array}{lllllll}
    \begin{array}{lll}
      \text{(Singleton Set)}\\
      \begin{array}{lll}
        [[{*} : 1 -> Set]]\\
        [[{*}]](\bullet) = [[{*}]]\\
        [[{*}]]([[id bullet]]) = [[id {*}]]
      \end{array}
    \end{array}
    & \quad & 
    \begin{array}{lll}
      \text{(Product Functor)}\\
      \begin{array}{lll}
        [[M * M : 1 -> Set]]\\
        [[(M * M)(bullet,bullet)]] = [[M(bullet) * M(bullet)]]\\
        [[(M * M)(id bullet,id bullet)]] = [[id (M(bullet) * M(bullet))]]\\
      \end{array}
    \end{array}
  \end{array}
  \]
  These are also both examples of set-valued functors.  The left
  functor is essentially the singleton set, and the right functor is
  the cartesian product of the underlying set of the monoid with
  itself.  Thus, a set-valued functor, $[[X : 1 -> Set]]$, is
  a set $[[X]]$ lifted up a level to functors.

\item Suppose we have a monoid $([[M]], [[e]], [[(x)]])$.
  A \emph{graded monoid} is a set-valued functor $[[G : M -> Set]]$ with two natural transformations:
  \[
  \begin{array}{lll}
    \text{(Identity)} & [[i : {*} -> G(e)]]\\
    \text{(Multiplication)} & [[ [(*)] : G(r1) * G(r2) -> G(r1 (x) r_2)]]
  \end{array}
  \]
  with the following commutative diagrams:
  \[
  \begin{array}{cccc}
    \begin{array}{lll}
      \bfig
      \square|amma|/->`->`=`->/<1000,500>[ [[G(r) * {*}]]`[[G(r)]]`[[G(r) * G(e)]]`[[G(r (x) e)]];[[pi1]]`[[id h{G(r)} * i]]``[[(*)]] ]
      \efig
      &
      \bfig
      \square|amma|/->`->`=`->/<1000,500>[ [[{*} * G(r)]]`[[G(r)]]`[[G(e) * G(r)]]`[[G(e (x) r)]];[[pi2]]`[[i * id h{G(r)}]]``[[(*)]] ]
      \efig    
    \end{array}
    \\[5pt]
    \bfig
    %% Top
    \morphism|a|/->/<1400,0>[ [[(G(r1) * G(r2)) * G(r3)]]`[[G(r1 (x) r2) * G(r3)]];[[ [(*)] * id h{G(r3)}]] ]
    \morphism(1400,0)|a|/->/<1200,0>[ [[G(r1 (x) r2) * G(r3)]]`[[G((r1 (x) r2) (x) r3)]];[[(*)]] ]

    %% Sides
    \morphism/=/<0,-500>[ [[(G(r1) * G(r2)) * G(r3)]]`[[G(r1) * (G(r2) * G(r3))]];]
    \morphism(2600,0)/=/<0,-500>[ [[G((r1 (x) r2) (x) r3)]]`[[G(r1 (x) (r2 (x) r3))]];]
    %% Bottom
    \morphism(0,-500)|a|/->/<1400,0>[ [[G(r1) * (G(r2) * G(r3))]]`[[G(r1) * G(r2 (x) r3)]];[[ id h{G(r1)} * [(*)] ]] ]
    \morphism(1400,-500)|a|/->/<1200,0>[ [[G(r1) * G(r2 (x) r3)]]`[[G(r1 (x) (r2 (x) r3))]];[[(*)]] ]
    \efig
  \end{array}
  \]
  The diagrams on top correspond to the identity axioms of the graded
  monoid, and the diagram on bottom corresponds to associativity of
  graded multiplication.

  An example graded monoid can be found in formal language theory.
  Take $\Sigma^n$ to be all the words of length $n$ over an alphabet
  $\Sigma$.  Then $\Sigma^0$ contains only the empty word, and can be
  lifted to a map $\varepsilon : [[{*}]] \mto \Sigma^0$, and
  composition of words lifts to a map $\cdot : \Sigma^m \times
  \Sigma^n \mto \Sigma^{m + n}$. Thus, since the empty word is the
  identity of composition of words, and composition is associative up
  to associativity of addition we have a graded monoid
  $(\Sigma^{-},\varepsilon, \cdot)$ over the additive natural number
  monoid $([[Nat]],0, +)$.  Using this graded monoid we can define the
  Kleene-star of an alphabet $\Sigma$ to be
  $\Sigma^* = \bigcup_{n \in [[Nat]]} \Sigma^n$.

\item There is one very important set-valued functor we have seen
  several times.  Suppose $A$ is an object of the category $\cat{C}$.
  Then we can define the \textbf{covariant hom-set functor}:
  \[
  \begin{array}{lll}
    \Hom{C}{A}{-} : \cat{C} \to \sets\\
    \Hom{C}{A}{B} = \{f : A \to B \mid f \in \mor{C}\}\\
    \Hom{C}{A}{h}(g) = g;h
  \end{array}
  \] 
  The proof that this is indeed a functor is left as an exercise. In
  addition, prove that this functor is natural in $B$.

\item Suppose $B$ is an object of the category $\cat{C}$.  Then we can
  define the \textbf{contravariant hom-set functor}:
  \[
  \begin{array}{lll}
    \Hom{C}{-}{B} : \catop{C} \to \sets\\
    \Hom{C}{A}{B} = \{f : A \to B \mid f \in \mor{C}\}\\
    \Hom{C}{f}{B}(g) = f;g
  \end{array}
  \] 
  The proof that this is indeed a functor is left as an exercise. In
  addition, prove that this functor is natural in both $A$.

\item Suppose $A$ and $B$ are objects of the category $\cat{C}$.  Then
  we can define the \textbf{hom-set functor}:
  \[
  \begin{array}{lll}
    \Hom{C}{-}{-} : \catop{C} \times \cat{C} \to \sets\\
    \Hom{C}{A}{B} = \{f : A \to B \mid f \in \mor{C}\}\\
    \Hom{C}{f}{h}(g) = f;g;h
  \end{array}
  \] 
  The proof that this is indeed a functor is left as an exercise. In
  addition, prove that this functor is natural in both $A$ and $B$.
\end{enumerate}
The last three examples are of particular importance, because they
underly the definition of a very important class of functors called
representable functors.
