One hotly debated question is, what is the essence of category theory?
First appearances would suggest composition as the answer, after all,
that's the defining property of a category.  However, if we think a
bit harder, and examine how we use categories we will find that simply
finding a category is not the most interesting use of our time, but
rather we find that we use categories to study constructions in them
or on them.  We strive to find relationships between these
constructions.  Consider the cartesian product as an example, of
which, we can find in two seemingly different categories:
\begin{itemize}
\item \textbf{Cartesian Products.} Given sets $A$ and $B$ their
  cartesian product $A \times B$ induces two functions $\pi_1 : A
  \times B \mto A$ and $\pi_2 : A \times B \mto B$ called the
  \textbf{projection maps}. Using these projections we can state a
  useful property of cartesian products.  Suppose we have functions $f
  : C \mto A$ and $g : C \mto B$ for some arbitrary set $C$.  Then
  there is a unique function $\langle f,g \rangle : C \mto A \times B$
  such that the following hold:
  \[
  \begin{array}{lll}
    \langle f,g \rangle;\pi_1 = f\\
    \langle f,g \rangle;\pi_2 = g
  \end{array}
  \]
  This property can be summed up as the following diagram:
  \begin{center}
    \begin{math}
      \bfig
      \dtriangle|ama|/->`-->`<-/<500,500>[C`A`A \times B;f`\langle f,g \rangle!`\pi_1]
      \btriangle(500,0)|maa|/-->`->`->/<500,500>[C`A \times B`B;\langle f,g \rangle!`g`\pi_2]
      \efig
    \end{math}
  \end{center}
  The dotted line stands for ``there exists'' and the bang (exclamation
  mark) stands for ``unique''.

  Notice that we have not given the definition of $\langle f,g \rangle$,
  but only its defining property, but this property
  implies that $\langle f,g \rangle$ must be defined to be:
  \[
  \begin{array}{lll}
    \langle f,g \rangle : C \mto A \times B\\
    \langle f,g \rangle(c) = (f(c),g(c))
  \end{array}
  \]
  Thus, this property gives us a way to pair functions with a common
  domain, and this pairing acts the way we expect it to act when we run
  it on some elements.

\item \textbf{Meets in Preorders.} Suppose we have a preorder
  $(P,\leq)$.  Then we say that the preorder $(P,\leq)$ has
  \textbf{meets} if given two elements $p,q \in P$, there is an
  element $p \land q \in P$ such that:
  \begin{enumerate}
  \item $(p \land q) \leq p$
  \item $(p \land q) \leq q$
  \item If $r \leq p$ and $r \leq q$, then $r \leq (p \land q)$
  \end{enumerate}
  Recall that if we treat the preorder $(P,\leq)$ as a category then
  we take $P$ as the set of objects, and there is a morphism $p \mto^f
  q$ if and only if $p \leq q$.  Thus, every homset $\Hom{P}{p}{q}$
  has at most one morphism in it.  Using this perspective we can
  recast the definition of meets in categorical terms.  It says, for
  any two objects $p$ and $q$ there is an object $p \land q$ with two
  projection maps $(p \land q) \mto^{\pi_1} p$ and $(p \land q)
  \mto^{\pi_2} q$ such that for any object $r$, if there are
  morphisms $r \mto^f p$ and $r \mto^g q$ there is a unique morphism
  $r \mto^{\langle f,g \rangle} p \land q$\footnote{Note here that if
    these morphisms exists then the latter is guaranteed to be unique,
    because at most one morphism exists between any two objects.}.

  Using this definition of meets we can prove two facts:
  \begin{enumerate}
  \item Given morphisms $r \mto^f p$ and $r \mto^g q$, $\langle f,g \rangle;\pi_1 = f$.
    \begin{proof}
      Suppose we have morphisms $r \mto^f p$ and $r \mto^g q$.  Thus,
      we know that $r \leq p$ and $r \leq q$.  Then we know by the
      definition of meets that there are morphisms $(p \land q)
      \mto^{\pi_1} p$ and $r \mto^{\langle f,g \rangle} p \land q$
      which both imply that $(p\land q) \leq p$ and $r \leq (p\land
      q)$.  We must show that $\langle f,g \rangle;\pi_1 = f$, but
      $\langle f,g \rangle;\pi_1$ is defined to be $r \leq (p \land q)
      \leq p$ which holds by transitivity, and is exactly $r \leq p$ or $r \mto^f p$.
    \end{proof}
    \item Given morphisms $r \mto^f p$ and $r \mto^g q$, $\langle f,g \rangle;\pi_2 = g$.
    \begin{proof}
      Left as an exercise.
    \end{proof}
  \end{enumerate}
  We can sum all of this up into the following diagram.  A preorder,
  $(P,\leq)$, has meets if for any two objects $p,q \in P$, there is
  an object $(p \land q) \in P$ such that the following diagram commutes:
  \begin{center}
    \begin{math}
      \bfig
      \dtriangle|ama|/->`-->`<-/<500,500>[r`p`p \land q;f`\langle f,g \rangle!`\pi_1]
      \btriangle(500,0)|maa|/-->`->`->/<500,500>[r`p \land q`q;\langle f,g \rangle!`g`\pi_2]
      \efig
    \end{math}
  \end{center}
  This is the same digram as the first example!  Thus, meets in a
  preorder have the same \textbf{defining property} as cartesian
  products in sets!
\end{itemize}
As we can see from the previous two examples cartesian products and
meets in a preorder share a common diagram that defines them. It is
quite surprising that these two seemingly different structures would
be defined in the same way, but as we have shown they both capture the
same structure, but in different categories.  A meet is a ``pair'' of
two elements in a preorder. The diagram
\begin{center}
  \begin{math}
    \bfig
    \dtriangle|ama|/->`-->`<-/<500,500>[C`A`A \times B;f`\langle f,g \rangle!`\pi_1]
    \btriangle(500,0)|maa|/-->`->`->/<500,500>[C`A \times B`B;\langle f,g \rangle!`g`\pi_2]
    \efig
  \end{math}
\end{center}
is the defining property of ``pairing'' objects and morphisms together
regardless of which category we are in.  If the category supports
pairing, then they will have a \textbf{product} object $A \times B$
and the previous diagram will hold. The implementation details of
pairing and each morphism in the previous diagram will be different in
different categories, but the property defined by the diagram will
remain the same. This is a very powerful realization, because this
shows that using this property we can relate different mathematical
structures.  Properties that allow us to do this, like the one for
products, are called \textbf{universal properties.}  Category theory
is chalked full of universal properties, because nearly every
categorical structure is defined by a universal property; and thus,
they are perhaps the most important tool in category.

In this section we will study universal properties focusing on their
precise definition.  This will require we understand two other
important tools in category called \textbf{Representable Functors} and
\textbf{The Yoneda Lemma}.  The road map we will follow is summarized
as follows:
\begin{itemize}
\item Introduce set-valued functors;
\item Introduce representable functors;
\item Introduce the Yoneda lemma;
\item Combine representable functors and the Yoneda lemma to introduce
  universal elements;
\item Define universal properties using universal elements;
\item Show universal properties are universal in the category of
  elements.
\end{itemize}
We will give lots of examples along the way.

One question might be, how do universal properties relate to
programming language theory?  This is a great question, and the answer
is, greatly!  Giving a semantics to a programming language requires
one to model the typing rules as representable functors, and the
operational semantics as the universal properties induced by these
representable functors through the Yoneda lemma.  We will conclude
this chapter with a detailed example for the simply typed
$\lambda$-calculus with products.  Then we will show how the Yoneda
lemma relates to normalization by evaluation a technique for deriving
termination algorithms for typed $\lambda$-calculi.


%%% Local Variables:
%%% mode: latex
%%% TeX-master: note.tex
%%% End:
