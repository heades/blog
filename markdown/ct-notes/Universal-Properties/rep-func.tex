Simply put, universal properties are what are called
\emph{representable functors.}  These correspond to set-valued
functors with a special property that states that they can be
completely captured by a collection of morphisms.  An example may help
make this informal explanation more clear.

Suppose $X$ is a set.  Then we can define the following functions
between $X$ and the set of functions $\Hom{[[Set]]}{[[{*}]]}{X}$ from
the singleton set $[[{*}]]$ to $X$:
\[
\begin{array}{lll}
  \begin{array}{lll}
    [[phi]] : [[X]] \mto \Hom{[[Set]]}{[[{*}]]}{X}\\
    [[phi]](x) = \lambda \_.x\\
  \end{array}
  &
  \begin{array}{lll}
    [[phi]]^{-1} : \Hom{[[Set]]}{[[{*}]]}{X} \mto [[X]]\\
    [[phi]]^{-1}(f) = f(\star)\\
  \end{array}
\end{array}
\]
The function $\phi$ is the constant function always returning $x$, and
its inverse takes in a function from $[[{*}]]$ to $X$, and simply
applies it to its only input.  We can easily see that this forms a
bijection.  Thus, every set $X$ is in bijection with the set of
functions $\Hom{[[Set]]}{[[{*}]]}{X}$.  Now consider the functor $[[id
    Set : Set -> Set]]$.  Using the above argument we can define a
natural isomorphism $\alpha : \Hom{[[Set]]}{[[{*}]]}{-} \mto [[id
    Set]]$ by setting $\alpha_X = [[phi]]^{-1}$.  Functors like $[[id
    Set]]$ that have such an object like $[[{*}]]$ that induces a
natural isomorphism like $\alpha$ are called \emph{representable
  functors}.

\begin{definition}
  \label{def:rep-funct}
  A set-valued functor $F : \cat{C} \mto \sets$ is called
  \textbf{representable} if and only if there is an object $A \in
  \Obj{C}$ and a natural isomorphism $\alpha : F \mto \Hom{C}{A}{-}$.
  We call the pair $(A,\alpha)$ a \textbf{representation} of $F$.
\end{definition}
The definition just given can be recast into a definition for
contravariant set-valued functors $F : \catop{C} \mto \sets$, called
presheafs, by proving it naturally isomorphic to the contravariant hom
functor.
\begin{definition}
  \label{def:rep-funct}
  A set-valued contravariant functor (presheaf) $F : \catop{C} \mto
  \sets$ is called \textbf{representable} if and only if there is an
  object $B \in \Obj{C}$ and a natural isomorphism $\beta : F \mto
  \Hom{\catop{C}}{-}{B}$.  We call the pair $(B,\beta)$ a
  \textbf{representation} of $F$.
\end{definition}

\ \\
\noindent
Let's consider some example representable functors.
\begin{enumerate}
\item The prototypical example of a representable functor is the
  powerset functor $\mathcal{P} : \sets^{\mathsf{op}} \mto \sets$
  defined by
  \[
  \begin{array}{lll}
    \mathcal{P}(X) & = & \{Y \mid Y \subseteq X\}\\
    \mathcal{P}(f : B \mto A) & = & \lambda X \subseteq A.\{x \in B \mid f(x) \in X\} : \mathcal{P}(A) \mto \mathcal{P}(B)\\
  \end{array}
  \]
  We should check that this defines a functor.  First we check that identities are preserved:
  \begin{center}
    \begin{math}
      \begin{array}{lll}
        \pow{\id : A \mto A}
        & = & \lambda X \subseteq A.\{x \in A \mid \id(x) \in X\}\\
        & = & \lambda X \subseteq A.\{x \in A \mid x \in X\}\\
        & = & \lambda X \subseteq A.X\\
      \end{array}
    \end{math}
  \end{center}
  Next we check that compositions are preserved.  Suppose $f \in
  \Hom{\sets^{\mathsf{op}}}{B}{A}$ and $g \in \Hom{\sets^{\mathsf{op}}}{C}{B}$.  Then
  \begin{center}
    \begin{math}
      \begin{array}{lll}
        \pow{g;f : C \mto A}
        & = & \lambda X \subseteq A.\{x \in C \mid f(g(x)) \in X\}\\
        & = & \lambda X \subseteq A.\{x \in C \mid f(g(x)) \in X\}\\
        & = & \lambda X \subseteq A.\{x \in C \mid g(x) \in \{y \in B \mid f(y) \in X\}\}\\
        & = & (\lambda Y \subseteq A.\{y \in B \mid f(y) \in Y\});(\lambda X \subseteq B.\{x \in C \mid g(x) \in X\})\\
        & = & \pow{f};\pow{g}
      \end{array}
    \end{math}
  \end{center}
  To prove that the powerset functor is representable we need to find
  a representation of it.  Thus, we must find a set $B$ together with
  a natural isomorphism $\alpha : \Hom{\sets^{\mathsf{op}}}{-}{B} \mto \mathcal{P}$.  The
  most difficult part is finding $B$.  To do this it is helpful to
  consider how $\mathcal{P}$ is defined.  Suppose we have such a $B$.
  Then we must define the component of $\alpha$, $\alpha_A :
  \Hom{\sets^{\mathsf{op}}}{A}{B} \mto \mathcal{P}(A)$, and the only thing we have to work
  with is the powerset functor.  To define the component $\alpha_A$ we
  are given a morphism $f : A \mto B$, and then the only sensible thing
  we can do is apply the powerset functor to it yielding $\pow{f} :
  \pow{B} \mto \pow{A}$, which by definition computes the inverse image
  of $f$ with respect to the input subset of $B$.  Now we need a
  subset of $A$, and the only way to get it is by finding a subset of
  $B$ in addition to $B$ itself.  Choose $B = \{0,1\}$, then we define
  $\alpha_A(f : A \mto B) = \pow{f}(\{1\})$.  This construction uses
  the characteristic function on $A$ to determine which subset to
  choose.

  It is easy to see that we can also define the component
  $\alpha^{-1}_A : \pow{A} \mto \Hom{\sets^{\mathsf{op}}}{A}{B}$ by
  \begin{displaymath}
    \alpha^{-1}_A(X)  = \lambda x \in A.\left\{
    \begin{array}{lr}
      1 & : x \in X\\
      0 & : x \notin X
    \end{array}
    \right.
  \end{displaymath}
  We can see that $\alpha_A;\alpha^{-1}_A = \id =
  \alpha^{-1}_A;\alpha_A$.  Hence, each $\alpha_A$ is an 
  isomorphism.

  Last but not least, we must show that $\alpha$ is a natural
  transformation.  Thus, we must prove that the following diagram
  commutes for any $f : A' \mto A$:
  \[\bfig
  \square<700,500>[\Hom{\sets^{\mathsf{op}}}{A}{B}`\pow{A}`\Hom{\sets^{\mathsf{op}}}{A'}{B}`\pow{A'};\alpha_A`\Hom{\sets^{\mathsf{op}}}{f}{B}`\pow{f}`\alpha_{A'}]
  \efig\] This follows from straightforward equational reasoning using
  the definitions of the respective functions in play.  

\item Suppose $A,B \in \Obj{C}$. What is a representation of the
  functor $\Hom{\cat{C}}{-}{A} \times \Hom{\cat{C}}{-}{B}$? It must be
  a pair $(C,\beta)$ such that $\beta : (\Hom{\cat{C}}{-}{A} \times
  \Hom{\cat{C}}{-}{B}) \mto \Hom{C}{-}{C}$ is natural isomorphism.
  The structure of $\beta$ tells us that given a pair $(f,g)$ of
  morphisms $f : X \mto A$ and $g : X \mto B$ we must produce a
  morphism $h : X \mto C$.  We have seen this pattern before in the
  universal property of products given in the introduction.  Suppose
  $\cat{C}$ has binary products, then take $C = A \times B$.  Then the
  universal property of binary products implies that $\beta :
  (\Hom{\cat{C}}{-}{A} \times \Hom{\cat{C}}{-}{B}) \mto \Hom{C}{-}{A
    \times B}$ is defined to be $\beta_{X}(f,g) = \langle f,g\rangle$.
  That is, the universal map of the binary product.  In the opposite,
  we have $\beta^{-1}_X(h) = (h;\pi_1,h;\pi_2)$.  The universal
  property of binary products implies that these are mutual inverses.

  The previous example assumes $\cat{C}$ has products, but
  representablity is more general than that.  It actually gives a
  formal means of describing the universal property of binary
  products.  Given a category $\cat{C}$ requiring that the functor
  $\Hom{\cat{C}}{-}{A} \times \Hom{\cat{C}}{-}{B}$ is representable
  implies there is a pair $(C,\beta)$ such that the following
  morphisms are definable:
  \[
  \begin{array}{lll}
    \pi_1 = \beta^{-1}_{C}(\id_C);\mathsf{fst} \in \Hom{\cat{C}}{C}{A}\\
    \pi_2 = \beta^{-1}_{C}(\id_C);\mathsf{snd} \in \Hom{\cat{C}}{C}{B}\\
    \langle f,g \rangle = \beta_{X}(f,g) \in \Hom{\cat{C}}{X}{C}\\
  \end{array}
  \]
  Now consider naturality of
  $\beta : \Hom{\cat{C}}{-}{A} \times \Hom{\cat{C}}{-}{B} \mto \Hom{C}{-}{C}$:
  \[
  \bfig
  \square|amma|<1500,1000>[
    \Hom{\cat{C}}{X}{A} \times \Hom{\cat{C}}{X}{B}`
    \Hom{C}{X}{C}`
    \Hom{\cat{C}}{Y}{A} \times \Hom{\cat{C}}{Y}{B}`
    \Hom{C}{Y}{C};
    \beta`
    \Hom{\cat{C}}{f}{\id_A} \times \Hom{\cat{C}}{f}{\id_B}`
    \Hom{\cat{C}}{f}{\id_C}`
    \beta]
  \efig
  \]
\end{enumerate}
