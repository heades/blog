
\begin{figure}
  \begin{center}
    \begin{math}
      \begin{array}{rrcll}
        \text{(Terms)} & [[t]], [[A]], [[B]], [[C]], [[D]]
         & ::= & [[Type l]]                   & \text{(Universe of Types)}\\
        &&  |  & [[( x : A ) -o B]]           & \text{(Dependent Linear Arrow)}\\
        &&  |  & [[( x : A ) * B]]            & \text{(Dependent Tensor Product)}\\
        &&  |  & [[x]]                        & \text{(Variable)}\\
        &&  |  & [[\x.t]]                     & \text{(Lambda Expression)}\\
        &&  |  & [[t1 t2]]                    & \text{(Function Application)}\\
        &&  |  & [[(t1 , t2)]]                & \text{(Linear Pair)}\\
        &&  |  & [[let ( x , y ) = t1 in t2]] & \text{(Linear Pair Eliminator)}\\
        \\
        \text{(Levels)} & [[l]] 
        & ::= & [[0]]                  & \text{(Level $0$)}\\
        && |  & [[1]]                  & \text{(Level $1$)}\\
        && |  & [[max(l1,l2)]]         & \text{(Maximum)}\\
        \\
        \text{(Typing Contexts)} & [[G]] 
        & ::= & [[.]]                  & \text{(Empty Context)}\\
        && |  & [[x:A]]                & \text{(Context Element)}\\
        && |  & [[G1,G2]]              & \text{(Context Extension)}\\
      \end{array}
    \end{math}
  \end{center}
  \caption{LDTT Expression Syntax}
  \label{fig:ldtt-syntax}
\end{figure}

\begin{figure}
  \begin{mathpar}
    \inferrule* [right=$\mathsf{Type}$] {
      [[l1 < l2]]
    }{[[. |- Type l1 : Type l2]]}
    \and
    \inferrule* [right=$[[-o]]$] {
      [[G1 |- A : Type l1]]\\
      [[G1 , G2 , x : A |- B : Type l2]]      
    }{[[G1 , G2 |- ( x : A ) -o B : Type max(l1,l2)]]}
    \and    
    \inferrule* [right=$[[*]]$] {
      [[G1 |- A : Type l1]]\\
      [[G1 , G2, x : A |- B : Type l2]]
    }{[[G1 , G2 |- ( x : A ) * B : Type max(l1,l2)]]}
    \and    
    \inferrule* [right=$\mathsf{Var}$] {      
      [[G |- A : Type l]]\\
      [[-| (G,x:A)]]
    }{[[G, x : A |- x : A]]}
    \and
    \inferrule* [right=$[[-o]]_i$] {
      [[G1,x : A |- B : Type l]]\\
      [[G1,G2,x : A |- t : B]]
    }{[[G1,G2 |- \x.t : (x : A) -o B]]}
    \and
    \inferrule* [right=$[[-o]]_e$] {
      [[G2 |- t1 : (y : A) -o B  ]]\\
      [[G1 |- t2 : A]]
    }{[[G1 , G2 |- t1 t2 : [t2/y]B]]}
    \and
    \inferrule* [right=$[[*]]_i$] {
      {
        \begin{array}{l}
          [[x nin FV(t2)]]\\
          [[G1 |- B1 : Type l1]]\\
          [[G1,G3,x : B1 |- B2 : Type l2]]
        \end{array}
      }\\
      {
        \begin{array}{l}
          \\
          [[G1,G2 |- t1 : B1]]\\
          [[G1,G3,G4,x : B1 |- t2 : B2]]
        \end{array}
      }
    }{[[G1,G2,G3,G4 |- (t1 , t2) : (x : B1) * B2]]}
    \and
    \inferrule* [right=$[[*]]_e$] {
      {
        \begin{array}{l}
          [[x,y nin FV(B3)]]\\
          [[G1 |- B1 : Type l1]]\\
          [[G1,G2,x : B1 |- B2 : Type l2]]
        \end{array}
      }\\
      {
        \begin{array}{l}
          \\
          [[G1,G2,G3 |- t1 : ( x : B1 ) * B2]]\\
          [[G1,G2,G4,x : B1,y : B2 |- t2 : B3]]
        \end{array}
      }
    }{[[G1,G2,G3,G4 |- let ( x , y ) = t1 in t2 : B3]]}
  \end{mathpar}
  \caption{LDTT Typing Rules}
  \label{fig:ldtt-typing-rules}
\end{figure}


\subsection{Metatheorems}
\label{subsec:metatheorems}
This section houses all of our metatheorems.  First, we consider any
two distinctly named contexts to be disjoint. Throughout the following
proofs we work up to dependent reordering of contexts.  That is, we do
not consider a typing derivation, $[[G1,x : A,y : B,G2 |- t : C]]$,
different from the typing derivation, $[[G1,y : B,x : A,G2 |- t :
    C]]$, when $[[x nin FV(B)]]$.

\subsubsection{Basic Metatheory}
\label{subsec:basic_metatheory}

\begin{lemma}[Inversion Principles]
  \label{lemma:inversion_principles}
  \begin{enumerate}[label=\roman*.]
  \item[] 
  \item If $[[G |- (x : A) -o B : Type l]]$, then $[[G1 |- A : Type l1]]$,
    $[[G1,G2, x : A |- B : Type l2]]$, and $[[l]] = [[max(l1,l2)]]$ for some contexts $[[G1]]$ and
    $[[G2]]$, and levels $[[l1]]$ and $[[l2]]$.
  \end{enumerate}
\end{lemma}

\begin{lemma}[Well-Formed Context Dependency Append]
  \label{lemma:well-formed_context_dependency_append}
  If $[[|- G1]]$ and $[[|- G2]]$, then $[[|- (G1,G2)]]$.
\end{lemma}

\begin{restatable}[Well-formed Context Dependency]{lemma}{wfxDepLem}
  \label{lemma:well-formed_context_dependency}
  If $[[G |- t : A]]$, then $[[|- G]]$.  
\end{restatable}

\begin{lemma}[Well-Formed Linear Contexts Append]
  \label{lemma:well-formed_linear_contexts_append}
  $[[-| G1]]$ and $[[-| G2]]$ iff $[[-| (G1,G2)]]$.
\end{lemma}

\begin{lemma}[Well-Formed Linear Contexts Extension]
  \label{lemma:well-formed_linear_contexts_extension}
  If $[[-| (G,x : A)]]$, then $[[-| G]]$.
\end{lemma}

\begin{restatable}[Well-formed Linear Context]{lemma}{wfxLinLem}
  \label{lemma:well-formed_context_linearity}
  If $[[G |- t : A]]$, then $[[-| G]]$.
\end{restatable}

\begin{corollary}[Well-Formed Contexts]
  \label{corollary:well-formed_contexts}
  If $[[G |- t : A]]$, then $[[G Ok]]$.
\end{corollary}

\begin{restatable}[Substitution for Typing]{lemma}{substLem}
  \label{lemma:substitution_for_typing}  
  Suppose
  $[[G2 |- A : Type l]]$,
  $[[G1,G2,x : A,G4 |- t2 : B]]$, and
  $[[G2,G3 |- t1 : A]]$.  Then
  $[[G1,G2,G3,[t1/x]G3 |- [t1/x]t2 : [t1/x]B]]$.
\end{restatable}

\begin{restatable}[Kinding for Typing]{lemma}{kindingLem}
  \label{lemma:kinding_for_typing}
If $[[G |- t : A]]$, then $[[G' |- A : Type l]]$ for some $[[G']] \subseteq [[G]]$ and level $[[l]]$.
\end{restatable}

\begin{lemma}[Arrow Kinding]
  \label{lemma:arrow_kinding}
  If $[[G |- t : (x : A) -o B]]$, then $[[G', x : A |- B : Type l]]$
  for some subcontext $[[G']] \subseteq [[G]]$ and level $[[l]]$.
\end{lemma}

% subsubsection basic_metatheory (end)

\subsubsection{Linearity}
\label{subsec:linearity}
\begin{definition}
  \label{def:disjointness}
  Suppose $[[G1 |- t : A]]$ and $[[G2]]$ is a second context.  Then we
  say $[[t]]$ and $[[G2]]$ are \emph{disjoint}, denoted $[[t >< G2]]$,
  if and only if $[[FV(t)]] \cap |[[G2]]| = \emptyset$.
\end{definition}

\begin{lemma}[Distributivity of Disjointness]
  \label{lemma:distributivity_of_disjointness}
  Suppose $[[t]]$ is a term, and that $[[t >< G]]$
  holds for some context $[[G]]$.  Furthermore, assume that the names
  in the domain of $[[G]]$ differ from the names of any bound
  variables in $[[t]]$.  Then by case analysis over the structure of
  $[[t]]$, the following distributivity properties hold:
  \begin{enumerate}[label=\roman*.]
  \item $[[((x : A) -o B) >< G]]$ if and only if $[[A >< G]]$ and $[[B >< G]]$
  \item $[[((x : A) * B) >< G]]$ if and only if $[[A >< G]]$ and $[[B >< G]]$  
  \item $[[(\x.t) >< G]]$ if and only if $[[t >< G]]$
  \item $[[(t1 t2) >< G]]$ if and only if $[[t1 >< G]]$ and $[[t2 >< G]]$
  \item $[[(t1,t2) >< G]]$ if and only if $[[t1 >< G]]$ and $[[t2 >< G]]$
  \item $[[(let (x,y) = t1 in t2) >< G]]$ if and only if $[[t1 >< G]]$ and $[[t2 >< G]]$
  \end{enumerate}
\end{lemma}

\begin{restatable}[Every Variable Must be Used]{lemma}{noWeakLem}
  \label{lemma:no_weakening}
  If $[[G |- t : A]]$, then for every
  $[[x : A in G]]$, $[[x in FV(G)]]$ or $[[x in FV(t)]]$ or $[[x in FV(A)]]$.  
\end{restatable}

\begin{lemma}
  \label{lemma:disjoint-contra}
  If $[[G |- t : A]]$ and $[[{t,A} >< G]]$, then $[[G]] = [[.]]$.
\end{lemma}

\begin{restatable}[Disjointness]{lemma}{disjointnessLem}
  \label{lemma:disjointness}
  If $[[G |- t : B]]$, $[[x : A in G]]$, $[[G1]] \subseteq [[G]]$, and $[[G1 |- A : Type l]]$, then
  $[[t >< G1]]$.
\end{restatable}

\begin{restatable}[Linearity]{theorem}{linThm}
  \label{thm:linearity}
  If $[[G |- t : B]]$, then for every $[[x : A in G]]$, $[[x]]$ appears only once in
  $[[G]]$, or only once in $[[t]]$, or only once in $[[B]]$.
\end{restatable}

\begin{corollary}
  \label{corollary:linearity}
  If $[[G,x : A,G' |- t : B]]$, then $[[x]]$ appears only once in
  $[[G']]$, or only once in $[[t]]$, or only once in $[[B]]$.
\end{corollary}
% subsubsection linearity (end)

\subsubsection{Trivialization}
\label{subsec:trivialization}

\begin{restatable}[$\lambda$-Bound Variables Must be Used]{lemma}{boundMustLem}
  \label{lemma:lambda-bound_variables_must_be_used}
  If $[[G |- \x.t : (x : A) -o B]]$, then $[[x in FV(t)]]$ and $[[x in FV(B)]]$.
\end{restatable}

\begin{lemma}[Closed Types]
  \label{lemma:closed_ctxs}
  If $[[G |- t : B]]$, $[[x : A in G]]$, $[[G1]] \subseteq [[G]]$, $[[G1 |- A : Type l]]$, and $[[B >< G1]]$,
  then $[[G1]] = [[.]]$.
\end{lemma}

\begin{lemma}[Arrow Bound Variables Must be Used]
  \label{lemma:arrow_bound_variables_must_be_used}
  If $[[G |- t : (x : A) -o B]]$, then $[[x in FV(B)]]$.
\end{lemma}

\begin{lemma}[No Type Dependency]
  \label{lemma:no_type_dependency}
  For any type $[[A]]$ and level $[[l]]$, there is no type $[[B]]$ such that
  $[[x : A |- B : Type l]]$.
\end{lemma}

\begin{restatable}[All Types are $\mathsf{Type}$]{lemma}{allTypesLem}
  \label{lemma:all_types_are_type}
  If $[[. |- A : Type l']]$, then $[[A]]$ is $[[Type l]]$ for some level $[[l]]$.
\end{restatable}

\begin{restatable}[Trivialization]{theorem}{trivThm}
  \label{theorem:trivalization}
  If $[[. |- t : A]]$, then $[[t]]$ is $[[Type l1]]$ and $[[A]]$ is
  $[[Type l2]]$ for some $[[l1]]$ and $[[l2]]$.
\end{restatable}

% subsubsection trivialization (end)
% subsection metatheorems (end)

%%% Local Variables:
%%% mode: latex
%%% TeX-master: spec-with-notes.tex
%%% End:
